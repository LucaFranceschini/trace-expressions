\section{Conclusion}
\label{sec:conclu}

We have presented the implementation of a prototype tool able to support
parametric runtime verification of Node.js applications in a distributed environment.

The tool is based on parametric trace expressions, and exploits Jalangi2 for tracing events via code instrumentation, and SWI-Prolog to
implement trace expressions.

Experiments have been conducted on \lstinline{http}, one of the most popular standard Node.js modules,
by mining from its documentation several specifications later expressed with trace expressions.

The mined specifications have been dynamically verified with the tool on several simple examples of HTTP servers and clients,
and with more interesting examples based on Express; these last tests required code instrumentation of the Express components
built on top \lstinline{http} for a total of almost 1K of SLOC, and have shown that the prototype tool is able to support
runtime verification of real Node.js code; the benchmarks show that the implemented optmizations allow a 120x speed boost for the more
complex Express example.
To our knowledge, no other currently available tool allows parametric runtime verification of Node.js applications in a distributed environment.

With respect to out previous work \cite{TowardsIoT17}, here we
\begin{enumerate*}[label=(\alph*)]
	\item presented a more complete set of examples, including HTTP interactions and the Express framework;
	\item enhanced our implementation in order to keep track of target objects in method invocations (necessary for monitoring HTTP interactions) and to deal with cyclic objects;
	\item optimized our system by filtering out events that are not relevant for the verified specification to avoid sending useless requests to the monitoring server, and by making monitoring asynchronous, to fit the Node.js execution model without suspending the execution of the program every time an event is observed.
	\item measure our performance with benchmarks.
\end{enumerate*}
More details can be found in the following subsections.

Although the reported experiments are promising, a considerable amount of work is still required for making our research prototype a tool usable in practice.

\subsection{Experiments}
Runtime verification of Express servers is an interesting result for the tool; unfortunately,
no benchmarking suites are available for further testing our tool, hence we are
considering to include in our suite other Express servers, either implemented by ourselves,
or freely available on the Web.

On the side of specification mining, there are still other interesting properties that can be expressed an dynamically
verified with parametric trace expressions which we are considering to mine both from the documentation of Node.js
\lstinline{http} module, and from the specification of HTTP. Furthermore, there is a number of other widely used Node.js
modules and frameworks as \lstinline{async} and \lstinline{jquery} for which specification mining and runtime
verification would be quite useful.

%% \subsection{Performance}
%% At the moment we have not focused on performance issues, because our first target was to have a proof of concept
%% that it is possible to implement a tool for parametric runtime verification for Node.js applications in a distributed environment.
%% As long as monitoring is used for software verification before deployment, problems with performance degradation are less 
%% stringent; however, the current implementation of our prototype still requires several optimizations to support runtime verification
%% in practice.

%% A first practical way to obtain a speed up of the monitored application is to
%% allow asynchronous communication between the instrumented code and the monitor;
%% indeed, the current implementation is based on synchronous communication, because this is
%% a simple way to guarantee that the monitor always receives events in the correct order.
%% Asynchronous communication is more complex to implement and can be achieved in different ways;
%% anyway, the speed up obtained by this solution depends on the responsiveness of the monitoring server,
%% which cannot be easily estimated without experimental data, because it depends on the specifically employed trace expression
%% for runtime verification.

%% Another useful optimization consists in reducing the network traffic 
%% between the monitored application and the monitor.
%% This can be effectively achieved in the following ways.

%% The current instrumentation implemented with Jalangi2 sends to the monitor
%% all events of the domain; in our case, all calls and returns from functions/callbacks.
%% However, the instrumentation can be configured so that it sends only those events 
%% of the domain which are relevant for the verification.

%% We have discussed in \Cref{sec:impl} that problems with getters and circular objects
%% prevented us to use \lstinline{JSON.stringify} to serialize values of the events to be sent to the
%% monitor. The customized version of \lstinline{stringify} we have implemented for the tool
%% can be optimized in several ways at the cost of a more complex code; for instance,
%% objects are always serialized at arbitrary depth (up to circularity, of course), but for
%% all kinds of verifications we have tested with our experiments, it is sufficient to serialize objects
%% at fixed depth not larger than 4 levels.

%% Finally, to optimize connection time, it would be worth sending sequence of events rather than one
%% event at time as happens in the current implementation.

\subsection{Error Reporting and IDE}
So far error reporting has been neglected, and in case of failure the only available information
regards the unexpected event that has been traced; clearly, more informative error messages
are required to allow the user to understand and locate more easily the detected problem.

Jalangi2 already supports information about the location of the original code that has been
instrumented, hence it should not be difficult to display the line of code and name of the file where
the faulty event has originated.

For error messaging, the SWI-Prolog implementation already supports identification of
all trace expressions that constitute a certain specification, hence associating specific error messages with them
would be a simple extension.
Moreover, the trace expression specification could be serialized and sent to the program for reporting purposes.
To this aim, JSON Schema \cite{jsonschema} can be useful.

Finally, at the moment, trace expressions must be written directly in Prolog and, although the used syntax
is similar to that defined in the paper, the interpreter adopts non standard rules to manage operator precedences,
and does not statically detect ill-formed trace expressions.

To this aim, we are developing in \textit{Xtext}\footnote{\url{https://eclipse.org/Xtext/}} an IDE
to write specifications with trace expressions at a higher level, and to automatically generate
the corresponding Prolog code. \textit{Xtext} allows such an IDE to be integrated as an Eclipse plugin
with support for syntax-highlighting, auto-completion,
and a number of static checks useful for early error detection of the written specification.
