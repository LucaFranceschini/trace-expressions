\section{Trace Expressions}
\label{sec:trace}
Trace expressions were initially developed in the context of multi-agent systems as ``global types'' \cite{AnconaDM12}.
The aim was to obtain a formalism expressive enough to check messages exchanged by autonomous agents for compliance against a given interaction protocol \cite{AnconaBFMT14, BriolaMA14}.
Global types, and thus trace expressions, derive from the notion of behavioral type, for which a survey can be found in \cite{AnconaBB0CDGGGH16}.

Later on, with the introduction of the notion of event type, trace expressions have become system and language agnostic, so that they can be exploited for runtime verification in several different contexts.

\paragraph{Events}
Trace expressions can be used to specify the possible correct behavior of a system, w.r.t.\ some property that needs to be verified.
To this aim, \emph{events} are defined as all the relevant observations that can be made on the system.
Examples includes the execution of I/O operations, the triggering of a Node.js callback, invocation of functions\textellipsis{} A fixed set \(\eventSet\) of events is assumed.

An \emph{event trace} \(\ev_1\dots \ev_n\dots\) is a (possibly infinite) sequence of events.
Using formal languages notation, the set of traces can be denoted as \(\eventSet^\infty = \eventSet^* \cup \eventSet^\omega\) (the union of all finite and infinite traces, respectively).
Intuitively a trace encodes a run of the system under test, or at least the relevant parts of its execution.

As an example, consider the monitoring of a program writing on files.
The set of relevant events could be, for instance, the three common operations \emph{open}, \emph{write} and \emph{close}, where \(\fd\) is the unique file descriptor:
%\[\mathcal{E} = \{ \mathit{open}(\mathit{fd}) \mid \mathit{fd} \in \mathbb{N} \} \cup
%\{ \mathit{write}(\mathit{fd}) \mid \mathit{fd} \in \mathbb{N} \} \cup
%\{ \mathit{close}(\mathit{fd}) \mid \mathit{fd} \in \mathbb{N} \} \]
\begin{equation}\label{eq:filedomain}
\eventSet = \bigcup_{\fd \in \mathbb{N}} \{\opent(\fd), \writet(\fd), \closet(\fd)\}
\end{equation}

An example of event trace over the set \(\eventSet\) would be the following:
\begin{equation} \label{eq:trace}
	\opent(42)\ \writet(42)\ \writet(42)\ \closet(42)
\end{equation}

\paragraph{Event Types}
On the top of events, a language \(\eventSet\) of \emph{event types} is defined.
Event types are generally terms allowed to contain variables, and their language is not fixed in order to make trace expressions more flexible and easily adaptable to different domains.

Together with event types a function \(\mtch\) is assumed to be given, with the following semantics.
Given an event \(\ev\) and an event type \(\eventTy\), \(\mtch(\ev, \eventTy) = \subs\) holds if and only if \(\ev\) matches \(\eventTy\) with the computed substitution \(\subs\); substitutions on terms have the usual meaning.

Considering again the previous files event domain (\ref{eq:filedomain}), a sensible event type matching all write operations would be \(\writet(\xv)\), where \(\xv\) is a variable.
A sensible definition of matching would lead to the following:
\[ \mtch(\writet(42), \writet(\xv)) = \{ \xv \mapsto 42 \} \]

%%Trace expressions have been extended with variables in \cite{AnconaFM17}.

\paragraph{Trace Expressions}
Finally, a \emph{trace expression} identifies a set of traces corresponding to correct system behaviors. It is built on top of event types and the following operators:
\begin{itemize}
	\item $\emptyseq$ (\emph{empty trace}): the singleton set $\{\emptyseq\}$ containing  the empty event trace $\emptyseq$;
	\item $\eventTy\prefixop\tau$ (\emph{prefix}): the set of all traces whose first event $\ev$ matches the event type $\eventTy$, and the remaining part is a trace of $\tau$;
	\item $\tau_1\catop\tau_2$ (\emph{concatenation}): the set of all traces obtained by concatenating the traces of $\tau_1$ with those of $\tau_2$; 
	\item $\tau_1\andop \tau_2$ (\emph{intersection}): the intersection of the traces of $\tau_1$ and $\tau_2$;
	\item $\tau_1\orop \tau_2$ (\emph{union}): the union of the traces of $\tau_1$ and $\tau_2$; 
	\item $\tau_1\shuffleop \tau_2$ (\emph{shuffle}, a.k.a. \emph{interleaving}): the set obtained by shuffling the traces of $\tau_1$ with the traces of $\tau_2$;
	\item $\var{x}{\tau}$ (\emph{binder}): it binds the free occurrences of $\xv$ in $\tau$;
	\item $\eventTy\filterop\tau$ (\emph{filter}):
	denoting the set of all traces contained in $\tau$, when they are deprived af all events that do not match $\eventTy$ (theoretically, this operator can also be derived from the others).
\end{itemize}

Trace expressions are regular terms (a.k.a. cyclic) \cite{Courcelle83}, thus there is no need for an explicit recursion operator.

For instance, the following trace expression \(\tau\) specifies the correct use of the file descriptor \(42\):
\begin{align*}
	\tau &= \emptyseq \orop (\opent(42) \prefixop \tau')\\
	\tau' &= (\writet(42) \prefixop \tau') \orop (\closet(42) \prefixop \emptyseq)
\end{align*}
In the first line, \(\opent\) is forced to be the first operation, if any.
After that, in \(\tau'\), either there will be write operations or the file will be closed and no more writes will be allowed.

%% Not all operators will be used in this document, but they all can be useful in different contexts.
%% See \cite{ancona2016comparing} for a complete technical presentation of trace expressions with more examples. 

\paragraph{Parametric Trace Expressions}
The trace expression above can only verify the correct use of a single file descriptor, but with variables and binders \cite{AnconaFM17} it is possible to write a \emph{parametric} specification that solves the problem for any number of files:
\begin{align}
\label{eq:files1}
\tau &= \emptyseq \orop \var{fd}{\opent(\avar{fd}) \prefixop (\tau \shuffleop \tau')}\\
\label{eq:files2}
\tau' &= (\writet(\avar{fd}) \prefixop \tau') \orop (\closet(\avar{fd}) \prefixop \emptyseq)
\end{align}
When the first event \(\opent(x)\) will occur, it will match the prefix computing the substitution \(\{\avar{fd}\mapsto x\}\).
At this point the shuffle operator is crucial\footnote{In order for the shuffle to work, we assume that every \(\opent\) operation always gives a fresh file descriptor, which can be reasonably assumed to be ensured by the operating system.}: following events are allowed to belong either to \(\tau\) (operations on new files, in the correct order) or to \(\tau'\{\avar{fd}\mapsto x\}\) (note the substitution!) were further operations on \(x\) will be checked.

\paragraph{Operational Semantics}
\begin{figure}[t]
\begin{gather*}
%%%% implicit substitution version, local substitution (i.e. substitution is output only)
\Rule{main}
{\tau\extrans{\ev}\tau';\emptyset}
{\tau\trans{\ev}\tau'}
{}
\qquad
\Rule{prefix}
{}
{\eventTy\prefixop\tau\extrans{\ev}\tau;\subs}
{
  \subs=\mtch(\ev,\eventTy)
}
\qquad
\Rule{and}
{\tau_1\extrans{\ev}\tau'_1;\subs_1\quad\tau_2\extrans{\ev}\tau'_2;\subs_2}
{\tau_1\andop\tau_2\extrans{\ev}\tau'_1\andop\tau'_2;\subs}
{\subs=\subs_1\subsMerge\subs_2}
\\[1ex]
\Rule{or-l}
{\tau_1\extrans{\ev}\tau'_1;\subs}
{\tau_1\orop\tau_2\extrans{\ev}\tau'_1;\subs}
{}
\qquad
\Rule{or-r}
{\tau_2\extrans{\ev}\tau'_2;\subs}
{\tau_1\orop\tau_2\extrans{\ev}\tau'_2;\subs}
{}
\qquad
\Rule{shuffle-l}
{\tau_1\extrans{\ev}\tau'_1;\subs}
{\tau_1\shuffleop\tau_2\extrans{\ev}\tau'_1\shuffleop\tau_2;\subs}
{}
\\[1ex]
\Rule{cat-l}
{\tau_1\extrans{\ev}\tau'_1;\subs}
{\tau_1\catop\tau_2\extrans{\ev}\tau'_1\catop\tau_2;\subs}
{}
\qquad
\Rule{cat-r}
{\tau_2\extrans{\ev}\tau'_2;\subs}
{\tau_1\catop\tau_2\extrans{\ev}\tau'_2;\subs}
{\isEmpty(\tau_1)}
\qquad
\Rule{shuffle-r}
{\tau_2\extrans{\ev}\tau'_2;\subs}
{\tau_1\shuffleop\tau_2\extrans{\ev}\tau_1\shuffleop\tau'_2;\subs}
{}
\\[1ex]
\Rule{var-t}
{\tau\extrans{\ev}\tau';\subs}
{\var{\xv}{\tau}\extrans{\ev}\subs\tau';\restrict{\subs}{\xv}}
{\xv\in\dom(\subs)}
\qquad
\Rule{var-f}
{\tau\extrans{\ev}\tau';\subs}
{\var{\xv}{\tau}\extrans{\ev}\var{\xv}{\tau'};\subs}
{\xv\not\in\dom(\subs)}
\\[1ex]
\Rule{filter-t}
{\tau\extrans{\ev}\tau';\subs}
{\eventTy \filterop \tau \extrans{\ev} \eventTy \filterop \tau';\subs}
{\subs=\mtch(\ev,\eventTy)}
\qquad
\Rule{filter-f}
{}
{\eventTy \filterop \tau \extrans{\ev} \eventTy \filterop \tau}
{\not\exists\subs=\mtch(\ev,\eventTy)}
\\[1ex]
\Rule{$\isEmpty$-empty}
{}
{\isEmpty(\emptyseq)}
{}
\qquad
\Rule{$\isEmpty$-var}
{\isEmpty(\tau)}
{\isEmpty(\var{\xv}{\tau})}
{}
\\[1ex]
\Rule{$\isEmpty$-or-l}
{\isEmpty(\tau_1)}
{\isEmpty(\tau_1\orop\tau_2)}
{}
\qquad
\Rule{$\isEmpty$-or-r}
{\isEmpty(\tau_2)}
{\isEmpty(\tau_1\orop\tau_2)}
{}
\qquad
\Rule{$\isEmpty$-others}
{\isEmpty(\tau_1)\quad\isEmpty(\tau_2)}
{\isEmpty(\tau_1 \op\ \tau_2)}
{\op\in\{\shuffleop,\catop,\andop\}}
\end{gather*}

\caption{Transition system for parametric trace expressions.}
\label{fig:semantics}
\end{figure}
The semantics of trace expressions is given by the labeled transition system in \Cref{fig:semantics}.
\( \tau \extrans{\ev} \tau';\subs \) holds iff the trace expression \(\tau\) accepts the event \(\ev\) with the substitution \(\subs\) and rewrites to \(\tau'\).
For instance, considering again trace expressions in \Cref{eq:files1,eq:files2}, the following transition is valid:
%%\[ \tau \extrans{\opent(1)} \tau';\{\avar{fd} \mapsto 1\} \]
\[
  \tau \trans{\opent(1)} \tau\shuffleop \tau'' \quad \mbox{ with }
  \tau''= (\writet(1) \prefixop \tau'') \orop (\closet(1) \prefixop \emptyseq)
\]

Transition rules depend on predicate \(\isEmpty(-)\) which checks for termination, i.e., \(\isEmpty(\tau)\) holds only if \(\tau\) accepts the empty trace \(\emptyseq\).
More generally, a trace expression \(\tau\) accepts a (possibly infinite) event trace \(\ev_1\ev_2\dots\) iff there exists a (possibly infinite) reduction \(\tau \trans{\ev_1} \tau' \trans{\ev_2} \dotsb\).

Trace expressions semantics has been implemented in SWI-Prolog.
The logic programming paradigm is well suited for inference systems implementation, since inference rules can be translated almost directly to logic clauses.
Furthermore SWI-Prolog offers native support to cyclic terms, therefore recursive trace expressions can be easily encoded.
Support for programming with cyclic terms is based on coinductive logic programming \cite{CoLP06}, which is supported by the SWI-Prolog library \texttt{coinduction}.
