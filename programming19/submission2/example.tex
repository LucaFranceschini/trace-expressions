\section{Monitoring Examples}
\label{sec:examples}
\subsection{Web Request Finalization}
The \lstinline{http} Node.js module exposes a complete API for HTTP clients and servers, and its typical uses are good examples of code that would benefit from runtime verification techniques.

Consider the snippet in \Cref{lst:end}: a HTTP server is created, and the given callback \lstinline|(req, res) => { ... }| defines its behavior, namely, it replies to every request with a response code \lstinline{200} and the content \lstinline{'okay'}.
The anonymous function will be invoked everytime a request is received from a client.

\begin{figure}[h]
\begin{lstlisting}
const http = require('http');
const server = http.createServer((req, res) => {
  res.writeHead(200);
  res.write('okay');
}); // res.end() call missing
server.listen(80);
\end{lstlisting}
\caption{An example of \emph{incorrect} usage of the Node.js \lstinline{http} module.}
\label{lst:end}
\end{figure}

It is possible for the server to send multiple chunks of data calling \lstinline{write}.
After that, the documentation \emph{requires} the programmer to call method \lstinline{end} to finalize the response.
The code above, however, fails to do that, making the client stuck waiting for the end of the response.
This constraint is not enforced by the library itself, and thus developers have to remember to call the \lstinline{end} method.

Trace expressions can be used to formally specify (and later verify) the requirement.
Consider for instance the following event domain:
\[ \eventSet = \bigcup_{\idcb,\idres \in \mathbb{N}} \{\createt(\idcb), \cbt(\idcb, \idres), \writet(\idres), \myend(\idres) \} \]

The four events above encodes calls to \lstinline{http.createServer}, the callback, \lstinline{res.write} and \lstinline{res.end}, respectively.
Every relevant objects has an associated unique identifier, both the callbacks and the response objects: this is needed in order to correctly match them.

We define event types to be event terms over a set of variables \vars used as placeholders for request identifiers (the match will simply compute the appropriate substitution):
\[ \vars = \{ \avar{id_1}, \avar{id_2}, \dotsc \}\qquad
\mathcal{ET} = \bigcup_{\xv,\xv' \in \vars} \{\createt(\xv), \cbt(\xv, \xv'), \writet(\xv), \myend(\xv) \} \]

Finally, the parametric trace expressions formalizing the expected trace of events follows:
\begin{gather*}
	\tau = \var{\idcb{}}{\createt(\idcb{}) \prefixop \tau_{cb}} \qquad\qquad
	\tau_{cb} = \var{\idres{}}{\cbt(\idcb, \idres) \prefixop (\tau_w \shuffleop \tau_{cb})}\\
	\tau_w = (\writet(\idres) \prefixop \tau_w) \orop (\myend(\idres) \prefixop \epsilon)
\end{gather*}

The trace must start with a \(\createt\) event, which also allows to store the identity of the the callback.
Then, once the callback is actually invoked, the identifier of the response object is known and it can later be used to match against subsequent \lstinline|write| and \lstinline|end| calls.

The use of the shuffle operator is crucial: it allows different requests to be handled concurrently, as this can happen when asynchronous computation is performed.

\subsection{Web Requests Length}
\label{sec:web-req}
Recalling the web server presented in \Cref{lst:http}, let us consider the (buggy) web client in \Cref{lst:client}, where seven bytes of data are sent (\lstinline{'hey!'}, \lstinline{'bye'}) but ten are declared in the \lstinline{'content-length'} field.
\lstinline{http.request} is an asynchronous operation that schedules the callback \lstinline!res => { ... }! for execution upon receivement of response.
Such callback will simply print the content of the message.

\begin{figure}[h]
\begin{lstlisting}
const http = require('http');
const options = {
	method: 'POST',
	headers: { 'content-length': 10 } };  // BUG! size too high
const req = http.request(options, res => res.on('data', console.log));
req.write('hey!');
req.end('bye');
\end{lstlisting}
\caption{An \emph{incorrect} web client sending a request with the wrong \texttt{content-length} value.}
\label{lst:client}
\end{figure}

The logging callback (\lstinline{console.log}) is scheduled for execution after receiving the response from the server.
However, it will not be called since the server will be waiting for more data, and no reply will be sent.

The \lstinline{'content-length'} is \emph{not} checked by the Node.js runtime against the actual length of the content, as it is clearly stated in the documentation.
Trace expressions can be used again to specify the correct behavior.
The monitor needs to observe the creation of the \lstinline{ClientRequest} object returned by function \lstinline{http.request} as well as the use of its methods \lstinline{write} and \lstinline{end}.

Consider for instance the following event domain:
\[ \eventSet = \bigcup_{\substack{\avar{data} \in \mathbb{B}^* \\ \avar{cl},\id \in \mathbb{N}}} \{ \req(\id,\avar{cl}), \writet(\id, \avar{data}), \myend(\id, \avar{data}) \} \]
All operations are associated with a unique request identifier (at the Node.js level, this corresponds to the \lstinline{ClientRequest} object), furthermore the request is associated with the declared \lstinline{'content-length'} \(\cl\), while \(\writet\) (and \(\myend\)) are associated with \(\data\),
the sequence of bytes being sent.

% lasciamo perdere, esplicitare event type qui e' un casino...

%Additionally, the language of event types will take into account the actual and expected lengths (a set of variables \(\vars\) is assumed):
%\[
%\mathcal{ET} =
%\bigcup_{\avar{id}, \avar{cl}, \avar{l}, \avar{l'} \in \vars}
%\{
%\req(\avar{id}, \avar{cl}),
%\writet(\avar{id}, \avar{l}, \avar{l'}),
%\myend(\avar{id, \avar{l}})
%\}
%\]

%The meaning of \(\writet(\avar{id}, \avar{l}, \avar{l'})\) is that before the write there were \(\avar{l}\) bytes left to be sent, and after that \(\avar{l'}\) (thus \(\avar{l'}\leq\avar{l}\)).
%The event type \(\myend(\avar{id}, \avar{l})\) corresponds to occurrences of event \(\myend(\id)\) with \(\avar{l}\) still to be sent.
%This will be encoded in the semantics of the \(\mtch\) function:
%\begin{align*}
%\mtch(\req(\id,\cl), \req(\avar{id}, \avar{cl})) &=
%\{ \avar{id} \mapsto \id, \avar{cl} \mapsto \cl \}\\
%\mtch(\writet(\id, \data), \writet(\avar{id}, \avar{l}, \avar{l'})) &=
%\\
%\mtch(\myend(\id), \myend(\avar{id, \avar{l}}))
%\end{align*}

The following is a possible trace expression specification of the correct behavior:
$$
\begin{array}{l}
\tau = \emptyseq \orop \var{id}{\var{cl}{ \req(\avar{id}, \avar{cl}) \prefixop (\tau_w\shuffleop\tau)}}\qquad
\tau_w = \var{cl'}{\writet(\avar{id}, \avar{cl}, \avar{cl'})\prefixop\tau_w'} \orop \tau_e\\
\tau_w' = \var{cl}{\writet(\avar{id}, \avar{cl'}, \avar{cl})\prefixop \tau_w} \orop \tau_e'\qquad
\tau_e = \myend(\avar{id}, \avar{cl})\prefixop \emptyseq\qquad
\tau_e' = \myend(\avar{id}, \avar{cl'})\prefixop\emptyseq
\end{array}
$$
The event types above have the following semantics:
\begin{itemize}
\item  \(\req(\avar{id}, \avar{cl})\): a request object identified by \(\avar{id}\) has been created, with a declared \lstinline{'content-length'} \(\avar{cl}\);
\item \(\writet(\avar{id}, \avar{cl}, \avar{cl'})\): \lstinline{write} method has been invoked on request \(\avar{id}\), with \(\avar{cl}\) bytes still to send, and after the invocation the amount of missing data is \(\avar{cl'}\);
\item \(\myend(\avar{id}, \avar{cl})\): \lstinline{end} has been invoked on request \(\avar{id}\), and the last \(\avar{cl}\) bytes are sent (besides finishing the request, \lstinline{end} can also send a final chunk of data).
\end{itemize}

Note that recursive occurrences are guarded by \(\writet\), thus forcing them to be alternating.

The shuffle operator, together with the unique identifier of each request, allows monitoring of multiple concurrent requests.
This is similar to the previous example on files.

Variables \(\avar{cl}\) and \(\avar{cl'}\) are swapped between \(\tau_w\) and \(\tau_w'\), as two variables are needed to keep track of the number of bytes left before and after each write.

Here the \(\mtch\) function plays a crucial role and it is worth showing, since in this case event types $\writet$ and $\myend$ are not a trivial abstraction over events:
\small
\begin{align*}
\mtch(\req(\id, \cl), \req(\avar{id}, \avar{cl})) &= \{ \avar{id} \mapsto \id, \avar{cl} \mapsto \cl \}\\
\mtch(\writet(\id, \data), \writet(\id', \cl, \avar{cl'})) &= \{ \avar{cl'} \mapsto \cl-\size(\data) \} \quad \mathrm{iff}\;\; \id=\id' \land \size(\data)\leq\cl \\
\mtch(\myend(\id, \data), \myend(\id', \cl)) &= \emptyset \quad \mathrm{iff}\;\; \id=\id' \land \size(\data)=\cl
\end{align*}
\normalsize

At the beginning, both variables \(\avar{id}\) and \(\avar{cl}\) are free and their values are discovered when event \(\req\) occurs.
On the other end, upon writing, the current amount of data that still has to be sent is known, but the updated one has to be computed based on the
size of $\data$.
Finally, \(\myend\) event is expected to send the exact amount of data left, and no further substitution is needed.

The \(\mtch\) function needs to be defined in order to use an event domain.
The same domain and \(\mtch\) function, however, can be used for many trace expressions.
Here, for instance, the function could be reused for any specification based on the event types above.
When event types are simply open terms over the signature of events, as it happens for instance in the first example on file operations in this section, the \(\mtch\) function is simply the standard term substitution computation.
