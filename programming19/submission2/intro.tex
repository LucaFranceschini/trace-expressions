\section{Introduction}
\emph{Node.js} is an increasingly popular runtime environment that supports server-side programming in JavaScript.
At its core, it is based on an asynchronous model of computation: I/O requests are non-blocking and they immediately return to the caller, non-deterministically scheduling the request for later execution.
Furthermore, while I/O operations can be parallelized by the underlying system, the JavaScript code always runs on a single thread.
This asynchronous, single-thread execution model turned out to be an effective way to handle big volumes of data and huge numbers of (simultaneous) requests \cite{Nodejs10,NodejsPerformance14}, thus making Node.js a suitable choice to implement efficient servers without explicitly dealing with multi-threading and parallelism.


%Besides, the huge package ecosystem promotes code reuse and fast development.
%Finally, Node.js recently gained popularity in the Internet of Things community as well.
%
%From a software engineering point of view however, Node.js has its downsides and pitfalls and poses serious challenges.

On the other hand, asynchronous programming heavily relies on the notion of callback, which can lead to many levels of nested blocks and obfuscate the actual execution flow, making the code hard both to read and to debug.
In reaction to an asynchronous request, a callback function is executed on the result of the corresponding operation when available.
This, together with the non-deterministic scheduling of requests, and the dynamic nature of JavaScript itself, are challenges to ensure correct program behavior with traditional approaches like static/formal verification techniques and testing.

\emph{Runtime verification} (RV) \cite{rv} is a software analysis approach in which a running system is observed by monitors that perceive relevant events and use their associated information to verify them against a given formal specification of the expected behavior.
When efficiently implemented, it can even support post production release verification.

As preliminary results suggest \cite{TowardsIoT17}, RV can be fruitfully used to monitor Node.js systems as a complement to formal verification and testing.
The approach suggested in \cite{TowardsIoT17} to implement a monitoring system for JavaScript and Node.js is based on code instrumentation, to allow suitable (and automatic) modification of the original code for appropriate reaction upon occurrences of interesting events.

For specifying systems behavior we use \emph{trace expressions} \cite{ancona2016comparing}, a formalism which is more expressive than others
commonly adopted for RV, as
attributed context-free grammars \cite{de2014combining} and LTL\textsubscript{3} \cite{ltl3}; see \cite{AnconaFM16} for a comparison between trace expressions and linear temporal logics.

Though trace expressions were initially devised to verify protocol-compliance in multi-agent systems, more recently they have been successfully
used as a more general specification formalism \cite{ParametricJava17, TowardsIoT17} thanks to their independence from the monitored system and underlying programming language, obtained through the abstraction of event type;
furthermore, they support \emph{parametric} verification, a key feature to increase the expressive power of the formalism and, hence, allow RV of complex properties whose specification may depend on the actual values associated with events at runtime.

In this work we show how trace expressions can be successfully used to support RV of Node.js applications, a topic that has not
been fully investigated yet;  to this aim, from the documentation of Node.js we have mined the main constraints that have
to be met to correctly use the functions exported by a module, and translated into trace expressions to allow RV of
such constraints.

A corresponding prototype%
%\footnote{The implementation and a script for running all tests reported in this paper are available at \url{https://github.com/LucaFranceschini/trace-expressions}.}
\footnote{Implementations and test at \url{https://github.com/LucaFranceschini/trace-expressions}.}
has been developed based on the Jalangi2\footnote{\url{https://github.com/Samsung/jalangi2}.}
framework and SWI-Prolog\footnote{\url{http://www.swi-prolog.org}.}.
The former is employed for instrumentation of JavaScript code, to be able
to dynamically trace all those events that constitute what we call an \emph{event domain}, that is, the set
of all events relevant to certain kinds of properties;
all examples shown in this paper refer to a unique event domain consisting of calls and returns
from asynchronous functions and their corresponding callbacks. Hence, a unique instrumentation
was required to automatically monitor and verify different properties, each specified by a particular trace expression.
SWI-Prolog is the underlying language for supporting parametric
trace expressions and implementing the inference engine for their verification.

The tool offers a simple HTTP interface based on JSON to allow RV of
distributed and heterogeneous components.
%; in this way, the tool is ready to support RV in the context of the Internet of Things.
%
To test the tool, we have inspected the documentation of the standard \lstinline{http} module, one of the most frequently used
in Node.js, to mine specifications expressible with trace expressions, with the aim of dynamically verifying
the correct use of \lstinline{http} in Node.js code.
Preliminary tests with Express, one of the most popular framework for web applications built on \lstinline{http}, show
that our prototype tool is able to support RV of real Node.js code.

The novel contributions w.r.t. our previous work \cite{TowardsIoT17} can be summarized as follows:
\begin{enumerate*}[label=(\alph*)]
	\item we present a more complete set of examples, including HTTP interactions and the Express framework;
	\item we enhance our implementation in order to keep track of target objects in method invocations (necessary for monitoring HTTP interactions) and to deal with cyclic objects;
	\item we optimize our system by filtering events that are not relevant for the verified specification to avoid sending useless requests to the monitoring server, and by making the communication between the monitored program and the monitor asynchronous.
          %% to fit the Node.js execution model without suspending the execution of the program every time an event is observed.
	\item we measure our performance with benchmarks.
\end{enumerate*}
%More details can be found in the following subsections.

%\vspace{-.3cm}
\paragraph{Outline}
\Cref{sec:node,sec:rv,sec:trace} give the necessary background and state of the art on Node.js, RV and trace expressions, respectively.
%\Cref{sec:node} motivates the choice of Node.js, and together with \Cref{sec:trace} they give the necessary background about Node.js and trace expressions.
%\Cref{sec:rv} is a study of the state of the art for RV techniques, while
\Cref{sec:examples} shows how the approach can help to detect bugs in Node.js programs with motivating examples.
\Cref{sec:impl} presents the implementation of our framework for RV of Node.js based on trace expressions.
\Cref{sec:exps} reports on some preliminary experiments conducted with Node.js programs based on the standard module
\lstinline{http} and the widely adopted framework Express.
Finally, \Cref{sec:conclu} draws conclusions and discusses future research directions.
