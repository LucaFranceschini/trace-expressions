\section{Implementation}
\label{sec:impl}
We have chosen to implement a monitoring system for Node.js based on \emph{code instrumentation}.
The core idea is to modify the original source code of the program to be monitored in a way that allows observation of relevant events.
Once an event is observed, the monitor can verify it against the specification, eventually acting in case of erroneous behavior.

The tool we have employed is \emph{Jalangi2} \cite{jalangi}.
% gia' linkato nelle prime pagine
%\footnote{\url{https://github.com/Samsung/jalangi2}} \cite{jalangi}
It allows addition of arbitrary code before and after basically every JavaScript operation: access to fields, declarations, functions and methods invocation, among others. Jalangi2 also gives access to information about the operation itself, as the arguments of a function call or the target object of a field access.
The tool allows for state-of-the-art performance in the context of code instrumentation for dynamic analysis.
To the best of our knowledge, the most similar tool is Linvail \cite{linvail}, though its performance are reported to be worst by an order of magnitude (w.r.t.\ the overhead).

Thanks to these features it is possible to determine when asynchronous calls and their callbacks are executed, as well as matching them.
However, implementing a prototype for Node.js runtime verification is not trivial; the instrumentation we had to implement
consists of more than 350 SLOC distributed over five files, and was required to address the following issues:
\begin{itemize}
\item Anonymous functions are extremely common in JavaScript, thus function names are not a reliable way to identify them.
\item Matching asynchronous functions with their callback needs some bookkeeping: the same asynchronous function can be used multiple times with different arguments and callbacks, and vice versa.
In order to deal with this problem, our prototype dynamically wraps callbacks \emph{at runtime} and generate a unique identifier for each one of them.
\item The prototype only tracks (i.e., instrument) the program, but does not directly interact with the Node.js runtime.
Since the execution goes back and forth between the event loop and the program, tracking the flow is not easy.
Still, our prototype is able to track all the calls, taking into account either the call site (when calling library functions) or the the instrumented body of the function being called (when callbacks are executed by the event loop).
Some non-trivial bookkeeping is needed in order to implement this, in order to avoid producing duplicate events when both information are available, that is, when the program is directly calling one of its functions.
\item Many different patterns arise in Node.js programming, and they need complex code instrumentation.
For instance, the file system module functions expect a file descriptor as an argument, while HTTP requests are handled through methods invoked on objects encoding the requests.
Jalangi2 partially solves this problem by exposing all relevant metadata, including the function to be called, the target object (in case of method invocation), the arguments and the return value.
Still, our prototype needs to store information about the target object in order to track subsequent method calls on the same object.
\end{itemize}
Note that even if we strongly rely on Jalangi2, these issues need to be faced with any code instrumentation framework when applied to Node.js code, because of the very nature of the framework and of JavaScript itself.

Our prototype implementation enriches the capabilities of Jalangi by supporting these additional features for Node.js monitoring.
We are currently looking at proxy-based alternative approaches based on reflection rather than instrumentation, since proxies \cite{proxy} are now natively supported by JavaScript.

The current implementation of trace expression semantics is coded in SWI-Prolog, since logic programming is a natural fit for inference rules and transition systems, and cyclic terms are supported by the \texttt{coinduction} library \cite{CoLP06}.

Our monitoring system has two main parts.
On one hand, an instrumented Node.js program is executed, and the additional code observes relevant events.
On the other hand, a Prolog server encoding the trace expression specification is running and validating events.
These two parts communicate through a HTTP interface, and events are exchanged in the JSON format.
The architecture is represented in \Cref{fig:arch}.
Further details on the approach and the implementation with Jalangi2 can be found in the related paper \cite{TowardsIoT17}.

Though JSON is a natural fit for JavaScript objects serialization, some important aspects need to be handled.
Objects containing circular references, for instance, are quite common in Node.js modules, but they cannot be encoded in JSON, and the JavaScript standard library serialization (\lstinline|JSON.stringify|) throws an error on such objects.
Furthermore, JavaScript supports getters, which are special functions invoked when a property is accessed.
During experiments, we found that some modules set getter functions before their code can actually be correctly executed, and this again breaks standard serialization since \lstinline|JSON.stringify| invokes getters during stringification.
For these reasons, we had to modify the JSON serialization process for it to work in real Node.js scenarios, detecting and handling circular references and getters as special cases.

Another challenge consisted in avoiding the performance penalties we detected with the first version of the
instrumenting code: the file explorer server employed in our benchmarks (see Section~\ref{sec:exps}) could serve no more than
three requests per second when monitored; that was an unacceptable result even when limiting
the use of runtime verification to the development stage for debugging and testing purposes.

A first practical way to obtain a speed up of the monitored application consisted in
avoiding the monitored application to directly synchronize with the monitoring Prolog server;
synchronous communication guarantees that the monitor always receives events in the correct order,
but has a negative effect on performances. To avoid this problem, we let the instrumented code
to asynchronously interact with a child process by sending to it all relevant events. Synchronous
communication with the Prolog  monitor is managed by the child process.

Another useful optimization consisted in reducing the network traffic 
between the monitored application and the monitor; while
the first version of the instrumentation was designed to send
all detected events, the optimization version sends only those events 
which are relevant for the verification.

%% This work is a preliminary step towards runtime verification of Internet of Things applications,
%% both because Node.js is emerging as a standard framework for IoT development, and the implementation through an HTTP
%% server offers a natural support to verification of distributed systems, where (instrumented) devices can send events to a monitoring server.

\begin{figure}
\centering\includegraphics[width=.7\textwidth]{fig/diagram}
\caption{Monitoring architecture for Node.js exploiting SWI-Prolog server with an HTTP interface.}
\label{fig:arch}
\end{figure}

