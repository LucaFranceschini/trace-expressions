\subsection{Benchmarks}
All tests presented in Section~\ref{sec:spec-minining} have been employed as benchmarks for performance evaluation
of our prototype on an Intel(R) Core(TM) i7-6500U CPU at 2.50GHz with a 16GB RAM, running SWI-Prolog 7.2.3 and Node.js v10.0.0 on
Linux kernel 4.14.69.
In particular, we have measured how much the performances of all implemented servers and clients are affected by runtime monitoring;
only correct implementations have been considered, to be able to run RV for arbitrary numbers of requests.
The benchmarks show that the overhead of runtime verification is mainly due to code instrumentation, while the Prolog specifications of the monitored properties (as defined in \Cref{sec:spec-minining}) have a lower impact.

\begin{table}[ht]
  \begin{tabular}{|l|l|l|l|}
    \hline
    \textbf{Benchmark} &
    \textbf{Original} &
    \textbf{Monitored} &
    \textbf{Overhead} \\
    \hline
    \multicolumn{4}{|c|}{HTTP module, server-side}\\
    \hline

    Omitted body (204 response)&
    716 RPS &
    660 RPS &
    8 \% \\

    Omitted body (304 response)&
    714 RPS &
    659 RPS &
    8 \% \\

    Write head &
    708 RPS &
    596 RPS &
    19 \% \\

    Response end &
    768 RPS &
    717 RPS &
    7 \% \\

    \hline

    \multicolumn{4}{|c|}{HTTP module, client-side}\\
    \hline

    Content length &
    697 RPS &
    541 RPS &
    29 \% \\

    Omitted body (HEAD request)&
    706 RPS &
    552 RPS &
    28 \% \\

    Unconsumed data &
     780 RPS &
     633 RPS &
     23 \% \\

     \hline
    \multicolumn{4}{|c|}{Express, server-side}\\
    \hline
    Hello world &
    767 RPS &
    744 RPS &
    3 \% \\

    File explorer &
    563 RPS &
    444 RPS &
    27 \%\\

    \hline
  \end{tabular}
  \caption{Benchmark results.}
  \label{table}
\end{table}
\vspace{-1em}
Table~\ref{table} shows the results of our experiments.
For each benchmark we have measured the average number of served requests per second (RPS) with clients running on the same machine a loop sending requests to the server without waiting to collect its response;
all measurements have been obtained as the average over ten different runs, each consisting of a loop of 100K requests.

All benchmarks consist of a pair of communicating Node.js server and client programs, where, depending on the specific
monitored property, either the server or the client has been monitored. Express benchmarks have been monitored only on the server side.
The second column (Original) displays the values concerning the original benchmark, while the third one
(Monitored) reports the measurements for the same benchmark under monitoring.
Except for the HTTP module client-side and the file explorer benchmarks the measured overhead is negligible;
for the other examples we have measured a modest overhead (below 29\%); we conjecture that the larger overhead
for these benchmarks is due to the higher number of events that have to be monitored.

For the Express file explorer presented in Section~\ref{sec:express}, the benchmarking client keeps sending GET requests
with randomly generated paths; in this case the overall performance of the monitored server is still acceptable, since it is able to operate at the rate of around 444 RPS; such a result shows that runtime monitoring can be effectively
employed for testing servers implemented with Express.
